%! TeX program = upLaTeX
\RequirePackage{plautopatch}
\documentclass[uplatex,dvipdfmx,fontsize=12pt,jafontsize=11pt,line_length=42zw,number_of_lines=36,hanging_punctuation]{jlreq}
\usepackage{jlreq-deluxe}
\usepackage{libertinus}
\usepackage[T1]{fontenc}
\usepackage{libertinust1math}
\usepackage{eucal}
\usepackage{mathnotes-jlreq}
\usepackage{tensor}

\pagestyle{empty}

\jlreqsetup{
	itemization_label_length = 2.5zw,
	itemization_labelsep = .5zw,
}
\setlength{\leftmargini}{3\zw}

\newcommand{\bdry}{\partial}
\newcommand{\abs}[1]{\lvert#1\rvert}
\newcommand{\compose}{\mathbin{\circ}}
\DeclareMathOperator{\Ric}{Ric}
\DeclareMathOperator{\id}{id}

\begin{document}

\begin{flushleft}
	幾何学4\,/\,微分幾何学概論I(松本)
	\hfill
	2022年12月12日
\end{flushleft}
\setcounter{section}{8}
\section{Hopf--Rinowの定理}

\begin{enumerate-problems}
	\item[9.1]
		$M=\mathbb{R}^2\setminus\set{0}$とおく.
		$\mathbb{R}^2$のEuclid計量を制限して得られる$M$のRiemann計量を$g$とする.
		\begin{enumerate-subproblems}
			\item
				$(M,g)$が完備でないことを3通りの方法で確かめよう.
				\begin{enumerate-alphabet}
					\item
						定義域を$[0,\infty)$まで延長できないような$(M,g)$の測地線の例を挙げよ.
					\item
						$(M,g)$の収束しないCauchy列の例を挙げよ.
					\item
						$(M,g)$の有界閉集合であってコンパクトでないようなものの例を挙げよ.
				\end{enumerate-alphabet}
			\item
				$p=(-1,0)$,$q=(1,0)$とおく.$(M,g)$において$d(p,q)=2$であることを示せ.
				また$p$から$q$に達する長さ$2$の区分的に滑らかな曲線が存在しないことを示せ.
		\end{enumerate-subproblems}
	\item[9.2]
		双曲平面$\mathbb{H}^2$は完備である.そのことを2通りの方法で確かめよう.
		\begin{enumerate-subproblems}
			\item
				$\mathbb{H}^2$は測地完備であることを示せ.
				[ヒント:測地線は問題7.2で記述してある.]
			\item
				$\mathbb{H}^2$の任意の有界閉集合はコンパクトであることを示せ.
				[ヒント:たとえばPoincar\'e開円板モデル$B^2$を用いる.
				双曲計量に関する有界閉集合$A$は,
				原点$0$を中心とする双曲計量に関するある距離球$B_r^d(0)$に含まれる.
				そこで$B_r^d(0)$が$B^2$のあるコンパクト部分集合に含まれることを確かめれば十分(なぜか?).]
		\end{enumerate-subproblems}
	\item[9.3]
		完備でないRiemann多様体であっても,
		「任意の2点$p$,$q$に対し$p$から$q$に達する最短測地線が存在する」という性質をみたす場合はある.
		例を挙げよ.
	\item[9.4]\phantom{}
		\begin{enumerate-subproblems}
			\item\vspace{-\baselineskip}
				Riemann多様体$(M,g)$では,
				距離球$B^d_r(p)$の閉包は$\set{q\in M|d(p,q)\leqq r}$にいつでも等しいことを示せ.
			\item
				一般の距離空間$(X,d)$では,
				開球$B^d_r(p)$の閉包は$\set{q\in X|d(p,q)\leqq r}$に等しいとは限らない.
				そのことを示す例を挙げよ.
		\end{enumerate-subproblems}
\end{enumerate-problems}

\end{document}
