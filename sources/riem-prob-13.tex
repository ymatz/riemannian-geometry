%! TeX program = upLaTeX
\RequirePackage{plautopatch}
\documentclass[uplatex,dvipdfmx,fontsize=12pt,jafontsize=11pt,line_length=42zw,number_of_lines=36,hanging_punctuation]{jlreq}
\usepackage{jlreq-deluxe}
\usepackage{libertinus}
\usepackage[T1]{fontenc}
\usepackage{libertinust1math}
\usepackage{eucal}
\usepackage{mathnotes-jlreq}
\usepackage{tensor}

\pagestyle{empty}

\jlreqsetup{
	itemization_label_length = 2.5zw,
	itemization_labelsep = .5zw,
}
\setlength{\leftmargini}{3\zw}

\newcommand{\bdry}{\partial}
\newcommand{\abs}[1]{\lvert#1\rvert}
\newcommand{\compose}{\mathbin{\circ}}
\DeclareMathOperator{\Ric}{Ric}
\DeclareMathOperator{\id}{id}

\begin{document}

\begin{flushleft}
	幾何学4\,/\,微分幾何学概論I(松本)
	\hfill
	2023年1月23日
\end{flushleft}
\setcounter{section}{12}
\section{Bonnet--Myersの定理}

\begin{enumerate-problems}
	\item[13.1]
		測地線$\gamma\colon[a,b]\to M$について,
		$\gamma$に沿って定義された区分的$C^\infty$級ベクトル場$V$,$W$に対しindex form
		\begin{equation}
			I(V,W)=\int_a^b(\braket{\dot{V},\dot{W}}+\braket{R(\dot{\gamma},V)\dot{\gamma},W})dt
		\end{equation}
		を考える($\dot{V}$は$\nabla_{\dot{\gamma}}V$を表す).
		$V$がJacobi場$J$であるとき
		\begin{equation}
			I(J,W)=\braket{\dot{J}(b),W(b)}-\braket{\dot{J}(a),W(a)}
		\end{equation}
		であることを確かめよ.
	\item[13.2]
		Rauchの比較定理を経由しないBonnetの定理の証明を与えよう.
		完備Riemann多様体$(M,g)$の断面曲率が$K(\sigma)\geqq\delta>0$をみたすとする.
		Hopf--Rinowの定理により,速さ$1$の測地線$\gamma\colon[0,T]\to M$は$T>\pi/\sqrt{\delta}$のとき
		最短測地線ではありえないことを証明すればよい.
		そのためには,$\gamma$に沿って定義された$V(0)=V(T)=0$をみたす区分的$C^\infty$級ベクトル場$V$であって,
		$I(V,V)<0$をみたすものが存在することを示せば十分である.

		$e$を$\gamma$に沿った長さ$1$の任意の平行法ベクトル場として
		\begin{equation}
			V(t)=\left(\sin\frac{\pi t}{T}\right)e(t)
		\end{equation}
		とおく.
		\begin{subproblems}
			\item
				$\displaystyle I(V,V)=\int_0^T\left(\sin^2\frac{\pi t}{T}\right)
				\left(\frac{\pi^2}{T^2}-\braket{R(e,\dot{\gamma})\dot{\gamma},e}\right)dt$を示せ.
			\item
				$I(V,V)<0$であることを結論せよ.よってBonnetの定理が従う.
		\end{subproblems}
	\item[13.3]
		前問の証明を改良してMyersの定理を示そう.
		$n$次元完備Riemann多様体$(M,g)$のRicciテンソルが$\Ric\geqq(n-1)\delta>0$をみたすとする.
		前問と同様に,速さ$1$の測地線$\gamma\colon[0,T]\to M$で$T>\pi/\sqrt{\delta}$なるものに対し,
		$\gamma$に沿って定義された$V(0)=V(T)=0$をみたす区分的$C^\infty$級ベクトル場$V$であって,
		$I(V,V)<0$をみたすものが存在することを示せば十分である.

		$e_1$,$e_2$,……,$e_{n-1}$を$\gamma$に沿った互いに直交する長さ$1$の平行法ベクトル場とし
		\begin{equation}
			V_i(t)=\left(\sin\frac{\pi t}{T}\right)e_i(t)
		\end{equation}
		とおく.
		$\displaystyle\sum_{i=1}^{n-1}I(V_i,V_i)<0$を示せ.
		よって$I(V_i,V_i)<0$をみたす$i$が少なくとも一つ存在するから,Myersの定理が従う.
\end{enumerate-problems}

\end{document}
