%! TeX program = upLaTeX
\RequirePackage{plautopatch}
\documentclass[uplatex,dvipdfmx,fontsize=12pt,jafontsize=11pt,line_length=42zw,number_of_lines=36,hanging_punctuation]{jlreq}
\usepackage{jlreq-deluxe}
\usepackage{libertinus}
\usepackage[T1]{fontenc}
\usepackage{libertinust1math}
\usepackage{eucal}
\usepackage{mathnotes-jlreq}
\usepackage{tensor}

\pagestyle{empty}

\jlreqsetup{
	itemization_label_length = 2.5zw,
	itemization_labelsep = .5zw,
}
\setlength{\leftmargini}{3\zw}

\newcommand{\bdry}{\partial}
\newcommand{\abs}[1]{\lvert#1\rvert}
\newcommand{\compose}{\mathbin{\circ}}
\DeclareMathOperator{\Ric}{Ric}
\DeclareMathOperator{\id}{id}

\begin{document}

\begin{flushleft}
	幾何学4\,/\,微分幾何学概論I(松本)
	\hfill
	2022年12月19日
\end{flushleft}
\setcounter{section}{9}
\section{Jacobi場}

\begin{enumerate-problems}
	\item[10.1]
		$F\colon I_1\times I_2\to M$を区間の直積$I_1\times I_2$からRiemann多様体への$C^\infty$級写像とし,
		曲線$\gamma_s\colon I_2\to M$を$\gamma_s(t)=F(s,t)$で,
		曲線$\omega_t\colon I_1\to M$を$\omega_t(s)=F(s,t)$で定義する.
		また各$(s,t)\in I_1\times I_2$に対し$V(s,t)\in T_{F(s,t)}M$が与えられており,
		$(s,t)$に関して$C^\infty$級であると
		する\footnote{$V$が接束$TM$への写像$I_1\times I_2\to TM$として$C^\infty$級ということ.
		$V(s,t)$を局所座標系を用いて表示したときの各成分が$(s,t)$に関して$C^\infty$級ということだと
		いってもよい.}.
		$\nabla_{\dot{\omega}_t}\nabla_{\dot{\gamma}_s}V-\nabla_{\dot{\gamma}_s}\nabla_{\dot{\omega}_t}V
		=R(\dot{\omega}_t,\dot{\gamma}_s)V$
		を示せ.[ヒント:局所座標系を用いて計算.]
	\item[10.2]
		測地線$\gamma\colon I\to M$に沿って定義されたJacobi場$J=J(t)$について,ある$t_0\in I$に対し
		$J(t_0)\perp\dot{\gamma}(0)$,$\dot{J}(t_0)\perp\dot{\gamma}(0)$ならば
		任意の$t\in I$に対し$J(t)\perp\dot{\gamma}(t)$であることを示せ.
		[ヒント:内積$\braket{J(t),\dot{\gamma}(t)}$を$t$に関して2回微分する.]
	\item[10.3]
		双曲平面$\mathbb{H}^2$におけるJacobi場の例を,
		測地線の1パラメタ族をもとに構成してみよう.
		上半平面モデル$H^2=\set{(x,y)\in\mathbb{R}^2|y>0}$を用いる.
		\begin{enumerate-subproblems}
			\item
				任意の実数$s$に対し,
				\begin{equation}
					\gamma_s(t)=\left(\frac{s(e^{2t}-1)}{1+s^2e^{2t}},\frac{(1+s^2)e^t}{1+s^2e^{2t}}\right)
				\end{equation}
				で与えられる$H^2$の曲線$\gamma_s$が測地線であることを確かめよ.
			\item
				$\gamma=\gamma_0$に沿ったJacobi場
				$J(t)=\left.\dfrac{\partial \gamma_s(t)}{\partial s}\right|_{s=0}$を考える.
				また,点$\gamma(0)=(0,1)$における接ベクトル$(1,0)\in T_{(0,1)}H^2$を平行移動して得られる
				$\gamma$に沿った単位平行ベクトル場を$e_2=e_2(t)$とする.
				$J(t)=(\sinh t)e_2(t)$を示せ.
			\item
				$J(t)$がJacobi方程式をみたすことを確かめよ.
		\end{enumerate-subproblems}
	\item[10.4]
		測地線$\gamma\colon[a,b]\to M$に沿って$\gamma(a)$と$\gamma(b)$が共役ではないと仮定する.
		そのとき,任意に与えられた$v\in T_{\gamma(a)}M$,$w\in T_{\gamma(b)}M$に対して,
		$J(a)=v$,$J(b)=w$をみたす$\gamma$に沿ったJacobi場$J(t)$がただ一つ存在することを示せ.
		[ヒント:$v=0$として証明すれば十分である(なぜか?).
		$J(a)=0$をみたすすべてのJacobi場がなすベクトル空間を$\mathcal{J}_0$とする.
		$J\mapsto J(b)$で定義される$\mathcal{J}_0$から$T_{\gamma(b)}M$への線型写像を考えよ.]
	\item[10.5]
		Riemann多様体$(M,g)$の\emph{Killingベクトル場}とは,
		その任意の局所フロー$\varphi_t\colon U\to M$($U$は開集合,$-\varepsilon<t<\varepsilon$)が
		$\varphi_t^*g=g|_U$をみたすようなもののことである.
		\begin{enumerate-subproblems}
			\item
				Killingベクトル場を測地線$\gamma$に制限したものはJacobi場であることを示せ.
			\item
				連結なRiemann多様体$(M,g)$では,Killingベクトル場$X$がある点$p\in M$において
				$X(p)=0$かつ$\nabla_v X=0$($v\in T_pM$は任意)をみたすならば,
				$M$全体で$X=0$であることを示せ.
		\end{enumerate-subproblems}
\end{enumerate-problems}

\end{document}
