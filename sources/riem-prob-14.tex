%! TeX program = upLaTeX
\RequirePackage{plautopatch}
\documentclass[uplatex,dvipdfmx,fontsize=12pt,jafontsize=11pt,line_length=42zw,number_of_lines=36,hanging_punctuation]{jlreq}
\usepackage{jlreq-deluxe}
\usepackage{libertinus}
\usepackage[T1]{fontenc}
\usepackage{libertinust1math}
\usepackage{eucal}
\usepackage{mathnotes-jlreq}
\usepackage{tensor}

\pagestyle{empty}

\jlreqsetup{
	itemization_label_length = 2.5zw,
	itemization_labelsep = .5zw,
}
\setlength{\leftmargini}{3\zw}

\newcommand{\bdry}{\partial}
\newcommand{\abs}[1]{\lvert#1\rvert}
\newcommand{\compose}{\mathbin{\circ}}
\DeclareMathOperator{\Ric}{Ric}
\DeclareMathOperator{\id}{id}

\begin{document}

\begin{flushleft}
	幾何学4\,/\,微分幾何学概論I(松本)
	\hfill
	2023年1月30日
\end{flushleft}
\setcounter{section}{13}
\section{Cartan--Hadamardの定理}

\begin{enumerate-problems}
	\item[14.1]
		完備Riemann多様体$(M,g)$が非正曲率をもつ(断面曲率が$K(\sigma)\leqq 0$をみたす)とする.
		そのとき$M$のどんな測地線$\gamma\colon[0,\infty)\to M$も
		始点$\gamma(0)$の共役点を$(0,\infty)$の範囲にもたないが,
		これは以下のようにして,Rauchの比較定理を使わずに簡単に証明できる.

		$\gamma$に沿って定義された$J(0)=0$,$\dot{J}(0)\not=0$をみたすJacobi場$J$を考える.
		仮定$K(\sigma)\geqq 0$を用いて$\dfrac{d}{dt}\abs{J(t)}^2\geqq 0$を示せ.
		またそのことから$t>0$で$\abs{J(t)}^2>0$であることを示せ.
	\item[14.2]
		$M$,$N$を$C^\infty$級多様体とし,$f\colon N\to M$を局所微分同相写像とする.
		この$f$が次に述べる\emph{smooth path lifting property}をもつと仮定する.
		\begin{quote}
			任意の$C^\infty$級曲線$\gamma\colon[0,T]\to M$と$f(q_0)=\gamma(0)$をみたす$q_0\in N$に対し,
			$q_0$を始点とする$C^\infty$級曲線$\tilde{\gamma}\colon[0,T]\to N$であって,
			$\gamma$のリフトになっている($f\circ\tilde{\gamma}=\gamma$をみたす)ようなものが存在する.
		\end{quote}
		そのとき$f$が可微分被覆写像であることを,以下に従って証明せよ.

		[注:位相空間のあいだの連続写像$f\colon Y\to X$についても同様の主張があるが,
		それが可微分カテゴリーでも成立するのだ,というのが本問の眼目である.]

		\begin{enumerate-subproblems}
			\item
				任意に$p\in M$をとる.$M$は多様体だから,$p$の弧状連結かつ単連結な開近傍$U$をとれる.
				$f^{-1}(U)$の連結成分への分解を
				$f^{-1}(U)=\bigsqcup_{\alpha\in A}V_\alpha$
				とする($f^{-1}(U)$は多様体$N$の開集合であることから局所弧状連結なので,$V_\alpha$は弧状連結でもある).
				各$V_\alpha$が$N$の開集合であることを示せ.
			\item
				各$V_\alpha$への$f$の制限$f|_{V_\alpha}\colon V_\alpha\to U$を考える.
				これが全単射であることを示せば,逆写像$(f|_{V_\alpha})^{-1}$の微分可能性は
				$f$が局所微分同相写像であることから直ちにわかるので,
				$f$が可微分被覆写像であると結論できる.

				まず,$f|_{V_\alpha}$が全射であることを,smooth path lifting propertyを用いて示せ.
			\item
				$f|_{V_\alpha}$の単射性を示そう.
				もし$f(q_1)=f(q_2)$,$q_1$,$q_2\in V_\alpha$ならば,
				$q_1$を出発し$q_2$に至る$V_\alpha$の$C^\infty$級曲線$\gamma\colon[0,T]\to V_\alpha$がある.
				$\underline{\gamma}=f\circ\gamma\colon[0,T]\to U$は$U$の閉曲線であり,
				$U$の単連結性により定曲線にホモトピック.
				このホモトピーを$F\colon[0,T]\times[0,1]\to U$とする(端点は固定しておく).
				$F$は$C^\infty$級写像とすることができる\footnote{本問はこの事実をとりあえず認めて解答してよい.
				Friedrichsの軟化子をうまく使えばよいのだが,
				写像$F$の値は多様体の点だから何らかの工夫が必要である.
				$U$を$\mathbb{R}^n$の原点を中心とする開球を像とするようなチャートとしておき,
				$F$を$\mathbb{R}^n$値関数とみなすのが一つの単純な方法だろう.}.

				$F$が$\tilde{F}(t,0)=\gamma(t)$をみたすリフト$\tilde{F}\colon[0,T]\times[0,1]\to V_\alpha$
				をもつことがわかれば$q_1=q_2$が従う.その理由を説明せよ.
			\item
				$0\leqq s_0<1$とする.$F$が仮に$[0,T]\times[0,s_0]$までは前述のようなリフト$\tilde{F}$をもつとしよう.
				$[0,T]$のコンパクト性に注意して,
				ある$\delta>0$が存在して$\tilde{F}$が$[0,T]\times[0,s_0+\delta)$における$F$のリフトに
				拡張することを示せ.
			\item
				$0<s_0\leqq 1$とする.
				$F$が$[0,T]\times[0,s_0)$において前述のようなリフト$\tilde{F}$をもつとしよう.
				そのとき,$\tilde{F}$は$[0,T]\times[0,s_0]$における$F$のリフトに拡張することを示せ.

				[ヒント:$\omega_{s_0}=F(t,s_0)$とおけば,$f$のsmooth path lifting propertyによって
				$\omega_{s_0}$は$q_1$を始点とするリフト$\tilde{\omega}_{s_0}\colon[0,T]\to N$をもつ.
				(4)と同様にして,$\tilde{\omega}_{s_0}$は
				$[0,T]\times(s_0-\delta,s_0]$における$F$のリフト$\tilde{F}'$に拡張する.
				ここで$s_0-\delta<s<s_0$に対しては,$F(\cdot,s)$の2通りのリフト
				$\tilde{F}(\cdot,s)$,$\tilde{F}'(\cdot,s)$が得られたことになるが,
				始点$\tilde{F}(0,s)$,$\tilde{F}'(0,s)$同士は一致することと$N$のHausdorff性に注意して,
				実は$\tilde{F}(\cdot,s)=\tilde{F}'(\cdot,s)$であることを示せ.
				したがって$\tilde{F}$と$\tilde{F}'$は定義域の共通部分で一致する.]
			\item
				(4)と(5)から,(3)で述べたようなリフト$\tilde{F}\colon[0,T]\times[0,1]\to V_\alpha$が
				存在することを結論せよ.
				ゆえに$f|_{V_\alpha}$は単射であり,$f$は可微分被覆写像であることがわかる.
		\end{enumerate-subproblems}
\end{enumerate-problems}

\end{document}
