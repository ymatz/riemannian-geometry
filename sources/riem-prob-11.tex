%! TeX program = upLaTeX
\RequirePackage{plautopatch}
\documentclass[uplatex,dvipdfmx,fontsize=12pt,jafontsize=11pt,line_length=42zw,number_of_lines=36,hanging_punctuation]{jlreq}
\usepackage{jlreq-deluxe}
\usepackage{libertinus}
\usepackage[T1]{fontenc}
\usepackage{libertinust1math}
\usepackage{eucal}
\usepackage{mathnotes-jlreq}
\usepackage{tensor}

\pagestyle{empty}

\jlreqsetup{
	itemization_label_length = 2.5zw,
	itemization_labelsep = .5zw,
}
\setlength{\leftmargini}{3\zw}

\newcommand{\bdry}{\partial}
\newcommand{\abs}[1]{\lvert#1\rvert}
\newcommand{\compose}{\mathbin{\circ}}
\DeclareMathOperator{\Ric}{Ric}
\DeclareMathOperator{\id}{id}

\begin{document}

\begin{flushleft}
	幾何学4\,/\,微分幾何学概論I(松本)
	\hfill
	2022年12月26日
\end{flushleft}
\setcounter{section}{10}
\section{曲線のエネルギーとその変分}

\begin{enumerate-problems}
	\item[11.1]
		$M$を完備Riemann多様体とし,$N$を$M$の閉部分多様体とする.
		$p$を$N$上にない$M$の点とする.
		$d(p,N)=\inf_{q'\in N}d(p,q')$とおく.
		\begin{enumerate-subproblems}
			\item
				$d(p,q)=d(p,N)$をみたす$q\in N$が存在することを示せ.
				[ヒント:Hopf--Rinowの定理により有界閉集合はコンパクト.]
			\item
				(1)のような$q$に対して$p$と$q$を結ぶ最短測地線$\gamma$を任意に一つとる.
				$\gamma$が$N$に直交することを示せ.
				[ヒント:曲線$\gamma$の最短性を直接利用してもよいが,ここではエネルギーを使う方針を示す.
				直交しないとする.
				そのとき始点$p$を固定する$\gamma$の変分$(\gamma_s)$で,変分ベクトル場$V$が
				$\braket{V(b),\dot{\gamma}(b)}<0$をみたすようなものをとれる.
				ある$s=s_0$が存在して$E(\gamma_{s_0})<E(\gamma)$であることを示し,
				$\gamma$が最短測地線であることと合わせて矛盾を導け.]
		\end{enumerate-subproblems}
	\item[11.2]\phantom{}
		\begin{enumerate-subproblems}
			\item\vspace{-\baselineskip}
				$V$を向きづけられた奇数次元実計量ベクトル空間とする.
				向きを保つ任意の直交変換$\varphi\colon V\to V$に対し,
				$\varphi(v)=v$をみたす$0$でないベクトル$v\in V$が存在することを示せ.
			\item
				$(M,g)$を向きづけられた偶数次元Riemann多様体とし,
				断面曲率が正である\footnote{各点$p\in M$における
				すべての2次元部分空間$\sigma\subset T_pM$に対し$K(\sigma)>0$であるという意味.}とする.
				任意の閉測地線$\gamma$に対し,$\gamma$がそれよりも短い閉曲線に
				ホモトピックであることを示せ\footnote{(なめらかな)閉曲線$\gamma\colon[a,b]\to M$とは,
				始点$\gamma(a)$と終点$\gamma(b)$が一致していて,
				しかも$t\in\mathbb{R}$に対して
				\begin{equation}
					\tilde{\gamma}(t)=\gamma(t-k(b-a)),
					\qquad\text{ただし$t\in\mathbb{Z}$は$t-k(b-a)\in[a,b]$となるように選ぶ}
				\end{equation}
				と定めたとき$\tilde{\gamma}\colon\mathbb{R}\to M$がなめらかな曲線であるようなもの.
				円周$S^1$からの$C^\infty$級写像のことだといってもよい.
				さらに上記の$\tilde{\gamma}$が測地線であるとき,閉曲線$\gamma$は(閉)測地線であるという.}.
				[ヒント:仮定と(1)により,閉測地線$\gamma$に沿って非自明な平行ベクトル場$V=V(t)$をとることができる.
				$V$を変分ベクトル場とする$\gamma$の変分$(\gamma_s)$をとる
				(ただし各$\gamma_s$も閉曲線とする).
				エネルギー$E(s)=E(\gamma_s)$について$E'(0)=0$,$E''(0)<0$を示し,利用せよ.]
		\end{enumerate-subproblems}
\end{enumerate-problems}

\end{document}
