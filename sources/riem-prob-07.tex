%! TeX program = upLaTeX
\RequirePackage{plautopatch}
\documentclass[uplatex,dvipdfmx,fontsize=12pt,jafontsize=11pt,line_length=42zw,number_of_lines=36,hanging_punctuation]{jlreq}
\usepackage{jlreq-deluxe}
\usepackage{libertinus}
\usepackage[T1]{fontenc}
\usepackage{libertinust1math}
\usepackage{eucal}
\usepackage{mathnotes-jlreq}
\usepackage{tensor}

\pagestyle{empty}

\jlreqsetup{
	itemization_label_length = 2.5zw,
	itemization_labelsep = .5zw,
}
\setlength{\leftmargini}{3\zw}

\newcommand{\bdry}{\partial}
\newcommand{\abs}[1]{\lvert#1\rvert}
\newcommand{\compose}{\mathbin{\circ}}
\DeclareMathOperator{\id}{id}

\begin{document}

\begin{flushleft}
	幾何学4\,/\,微分幾何学概論I(松本)
	\hfill
	2022年11月28日
\end{flushleft}
\setcounter{section}{6}
\section{測地線(1)}

\begin{enumerate-problems}
	\item[7.1]
		球面$S^n$の測地線について調べる.
		\begin{enumerate-subproblems}
			\item
				$\mathbb{R}^{n+1}$の互いに直交する2つの単位ベクトル$a$,$b$を用いて
				\begin{equation}
					\gamma_{a,b}(t)=(\cos t)a+(\sin t)b,\qquad t\in(-\infty,\infty)
				\end{equation}
				と定める.
				$\gamma_{a,b}$が$S^n$の測地線であることを示せ.[ヒント:問題5.4を利用せよ.]
			\item
				速さ$1$の測地線$\gamma\colon I\to S^n$は
				ある$a$,$b$を用いて$\gamma(t)=\gamma_{a,b}(t)$と表されることを示せ.
				[ヒント:$0\in I$と仮定して問題ない.
				そして測地線は$\gamma(0)$,$\dot{\gamma}(0)$を指定すれば一意的.]
		\end{enumerate-subproblems}
	\item[7.2]
		双曲平面$\mathbb{H}^2$の測地線について調べる.
		上半平面モデル$H^2=\set{(x,y)\in\mathbb{R}^2|y>0}$を用いることにしよう.
		Riemann計量は$g=y^{-2}(dx^2+dy^2)$で与えられる.
		\begin{enumerate-subproblems}
			\item
				$\gamma_0(t)=(0,e^t)$で定義される曲線
				$\gamma_0\colon(-\infty,\infty)\to H^2$が測地線であることを示せ.
				[ヒント:問題3.2.]
			\item
				点$(x,y)$を複素数$z=x+iy$と同一視する.
				次の3種類の写像がそれぞれ$H^2$の等長変換を与えることを示せ.
				\begin{enumerate-alphabet}
					\item $\Phi(z)=\lambda z$($\lambda$は正の実数).
					\item $\Phi(z)=z+c$($c$は実数).
					\item $\Phi(z)=-1/z$.
				\end{enumerate-alphabet}
			\item
				速さ$1$の測地線$\gamma\colon I\to H^2$は
				ある等長変換$\Phi$によって
				$\gamma=\Phi\compose\gamma_0$と表されることを示せ.

				[ヒント:長さ$1$の任意の接ベクトル$v\in T_{(0,1)}H^2$に対し,
				$\gamma(0)=(0,1)$,$\dot{\gamma}(0)=v$をみたす測地線$\gamma$を$\Phi\compose\gamma_0$の
				形で表せることを示せば十分である(なぜか?).
				任意の実数$c$に対し,(2)で挙げた3種類の写像を合成して得られる
				\begin{equation}
					z\mapsto z+c
					\mapsto -\frac{1}{z+c}
					\mapsto -\frac{c^2+1}{z+c}
					\mapsto -\frac{c^2+1}{z+c}+c=\frac{cz+1}{z+c}
				\end{equation}
				も等長変換であることを利用せよ.]
		\end{enumerate-subproblems}
		\clearpage
	\item[7.3]\phantom{}
		\begin{enumerate-subproblems}
			\item\vspace{-\baselineskip}
				$(M,g)$を連結なRiemann多様体とし,$\Phi\colon M\to M$を等長変換とする.
				ある点$p\in M$において$\Phi(p)=p$,$(d\Phi)_p=\id_{T_pM}$ならば
				$\Phi$は恒等変換であることを示せ.
				[ヒント:$A=\set{q\in M|\Phi(q)=q,\,(d\Phi)_q=\id_{T_qM}}$とおく.
				仮定によって$A$は空集合ではない.$A$が開集合かつ閉集合であることを示す.]
			\item
				Euclid空間$\mathbb{R}^n$の等長変換が
				$\Phi(x)=Ax+b$($A$は$n$次直交行列,$b$は$\mathbb{R}^n$のベクトル)の形に表されるものに
				限られることを示せ.
			\item
				球面$S^n$の等長変換が$\mathbb{R}^{n+1}$の直交変換の制限によって得られるものに
				限られることを示せ.
		\end{enumerate-subproblems}
	\item[7.4]
		$\gamma\colon I\to M$を速さ$1$の測地線とする(ただし$I\subset\mathbb{R}$は$0$を含む区間).
		$p=\gamma(0)$において
		$\dot{\gamma}(0)$,$v_2$,……,$v_n$が$T_pM$の正規直交基底となるように
		$n-1$個のベクトル$v_j$($2\leqq j\leqq n$)をとる.
		各$j$に対し,$e_j(t)$を$e_j(0)=v_j$をみたす$\gamma$に沿って定義された平行ベクトル場とする.
		すると各$t\in I$において$\dot{\gamma}(t)$,$e_2(t)$,……,$e_n(t)$は$T_{\gamma(t)}M$の正規直交基底である.

		逆に,速さ$1$の曲線$\gamma\colon I\to M$に沿って平行ベクトル場$e_2(t)$,……,$e_n(t)$が定義されており,
		各$t\in I$に対し$\dot{\gamma}(t)$,$e_2(t)$,……,$e_n(t)$が$T_{\gamma(t)}M$の正規直交基底であるとする.
		そのとき$\gamma$は測地線であることを示せ.
	\item[7.5]
		Riemann多様体$(M,g)$の点$p\in M$において正規球$B_\varepsilon(p)$をとり,
		この正規球における正規座標系$(x^1,\dots,x^n)$を考える.
		\begin{enumerate-subproblems}
			\item
				$(x^1,\dots,x^n)$に関するChristoffel記号$\tensor{\Gamma}{^k_i_j}$の点$p$における値が
				すべて$0$であることを示せ.
				[ヒント:任意の$a=(a^1,\dots,a^n)\in\mathbb{R}^n$に対して
				$\gamma(t)=(a^1t,\dots,a^nt)$は測地線である.これを測地線の方程式に代入する.]
			\item
				$(x^1,\dots,x^n)$に関して$(\partial\tensor{g}{_i_j}/\partial x^k)(p)=0$であることを示せ.
		\end{enumerate-subproblems}

		(実は$\tensor{g}{_i_j}$の2次までのTaylor展開は
		\begin{equation}
			\tensor{g}{_i_j}=\tensor{\delta}{_i_j}-\frac{1}{3}\tensor{R}{_i_k_j_l}(p)x^kx^l+O(\abs{x}^3)
		\end{equation}
		で与えられる(第12回?).
		つまりRiemann曲率テンソルは,正規座標系における$g$とEuclid計量のずれの主要部を表している.)
\end{enumerate-problems}

\end{document}
