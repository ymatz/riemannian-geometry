%! TeX program = upLaTeX
\RequirePackage{plautopatch}
\documentclass[uplatex,dvipdfmx,fontsize=12pt,jafontsize=11pt,line_length=42zw,number_of_lines=36,hanging_punctuation]{jlreq}
\usepackage{jlreq-deluxe}
\usepackage{libertinus}
\usepackage[T1]{fontenc}
\usepackage{libertinust1math}
\usepackage{eucal}
\usepackage{mathnotes-jlreq}
\usepackage{tensor}

\pagestyle{empty}

\jlreqsetup{
	itemization_label_length = 2.5zw,
	itemization_labelsep = .5zw,
}
\setlength{\leftmargini}{3\zw}

\newcommand{\bdry}{\partial}
\newcommand{\abs}[1]{\lvert#1\rvert}
\newcommand{\compose}{\mathbin{\circ}}
\DeclareMathOperator{\Ric}{Ric}
\DeclareMathOperator{\id}{id}
\DeclareMathOperator{\Vol}{Vol}
\DeclareMathOperator{\Area}{Area}

\begin{document}

\begin{flushleft}
	幾何学4\,/\,微分幾何学概論I(松本)
	\hfill
	2023年1月16日
\end{flushleft}
\setcounter{section}{11}
\section{Rauchの比較定理}

\begin{enumerate-problems}
	\item[12.1]
		ある種の状況下では,Rauchの比較定理から得られる帰結は比較的簡単に直接示せる.

		例として,本問では非正曲率をもつ$(M,g)$を考える
		(断面曲率が各点で任意の$\sigma$に対して$K(\sigma)\leqq 0$をみたすとする).
		そのときRauchの比較定理を用いてEuclid空間と比較することにより,
		任意の測地線$\gamma\colon[0,T]\to M$および
		$\gamma$に沿った法Jacobi場$J$で$J(0)=0$,$\abs{\dot{J}(0)}=1$をみたすものに対し
		($\dot{J}$は$\nabla_{\dot{\gamma}}J$のこと),
		$0\leqq t\leqq T$で$\abs{J(t)}\geqq t$が成り立つことが従う.
		とくに$t>0$で$J(t)\not=0$だが\footnote{このことから$\gamma$上に
		点$\gamma(0)$の共役点が存在しないことも直ちにわかる.これは任意のJacobi場は
		接成分と法成分に分解しても各々の成分がJacobi場であり,
		しかも接Jacobi場とは$(a+bt)\dot{\gamma}(t)$という形($a$,$b\in\mathbb{R}$)のベクトル場にすぎないからである.
		次回触れる.},
		これを直接証明してみよう.
		\begin{enumerate-subproblems}
			\item
				$f(t)=\abs{J(t)}^2$とおく.これは$C^\infty$級関数である
				(局所座標系を用いれば$J(t)$の各成分は$t$について$C^\infty$級だから
				\footnote{もし$f(t)=\abs{J(t)}$としてしまうと,
				$J(t)=0$となる$t$において$f(t)$が微分可能でない恐れが生じる.}).
				$f'(t)=2\braket{J(t),\dot{J}(t)}$,
				$f''(t)=2\abs{\dot{J}(t)}^2+2\braket{J(t),\ddot{J}(t)}$
				である.これと仮定$K(\sigma)\leqq 0$を用いて$f''(t)\geqq 0$を示せ.
			\item
				$f(0)=f'(0)=0$,$f''(0)=1$に注意して,$t>0$ならば$f(t)>0$である
				(したがって$J(t)\not=0$である)ことを示せ.
		\end{enumerate-subproblems}
	\item[12.2]
		$\gamma\colon[a,b]\to M$を測地線とする.
		$(\gamma_s)_{-\varepsilon<s<\varepsilon}$を$\gamma$の変分であって両端を固定するようなものとするとき,
		変分ベクトル場を$V$とすれば
		\begin{equation}
			\left.\frac{dE(\gamma_s)}{ds}\right|_{s=0}=0,\qquad
			\left.\frac{d^2E(\gamma_s)}{ds^2}\right|_{s=0}=I(V,V)
		\end{equation}
		であった.

		実は,$(\gamma_s)$が両端を固定するような変分でなかったとしても,
		$s\mapsto\gamma_s(a)$,$s\mapsto\gamma_s(b)$がそれぞれ測地線になっていれば同じ等式が成り立つ.
		それを確かめよ.
		ただしここでは
		$\displaystyle I(V,V)=\int_a^b(\abs{\nabla_{\dot{\gamma}}V}^2+\braket{R(\dot{\gamma},V)\dot{\gamma},V})dt$
		をindex form $I(V,V)$の定義とする.
	\clearpage
	\item[12.3]
		$(M,g)$,$(\tilde{M},\tilde{g})$はRiemann多様体で$\dim M\leqq \dim\tilde{M}$とする.
		任意の$p\in M$,$\sigma\subset T_pM$と
		任意の$\tilde{p}\in\tilde{M}$,$\tilde{\sigma}\subset T_{\tilde{p}}\tilde{M}$について
		断面曲率の値が$K(\sigma)\geqq K(\tilde{\sigma})$をみたすと仮定する.

		あらためて点$p\in M$,$\tilde{p}\in\tilde{M}$をとり固定する.
		正の数$r$を,点$p$における指数写像$\exp_p$が$B_r(0)\subset T_pM$で定義されて臨界点をもたず,
		また$\exp_{\tilde{p}}$も$B_r(0)\subset T_{\tilde{p}}\tilde{M}$で定義されるようにとっておく
		\footnote{本問はCheeger \& Ebin, \textit{Comparison Theorems
		in Riemannian Geometry}の系1.35やdo Carmo, \textit{Riemannian Geometry}の第10章命題2.5にあたるが,
		これらの文献は$r$にもっと強い制約を課している.ここに書いたことで十分なように思えるのだがどうでしょうか.
		修正が必要であればそれも検討して,私に教えてください.}.
		\begin{enumerate-subproblems}
			\item
				$0<T<r$として,$\gamma\colon[0,T]\to M$,$\tilde{\gamma}\colon[0,T]\to\tilde{M}$を
				それぞれ$p$,$\tilde{p}$を始点とする速さ$1$の測地線とする.
				$v\in T_pM$,$\tilde{v}\in T_{\tilde{p}}\tilde{M}$であって
				$\abs{v}=\abs{\tilde{v}}$,
				$\braket{\dot{\gamma}(0),v}=\braket{\dot{\tilde{\gamma}}(0),\tilde{v}}$
				をみたすものに対し,
				$0\leqq t\leqq T$について
				$\abs{(d\exp_p)_{t\dot{\gamma}(0)}(v)}\leqq\abs{(d\exp_{\tilde{p}})_{t\dot{\tilde{\gamma}}(0)}(\tilde{v})}$
				であることを示せ.
				[ヒント:$J(t)=(d\exp_p)_{t\dot{\gamma}(0)}(tv)$は$\gamma$に沿ったJacobi場で
				$J(0)=0$,$\dot{J}(0)=v$である.
				$\tilde{J}(t)=(d\exp_{\tilde{p}})_{t\dot{\tilde{\gamma}}(0)}(t\tilde{v})$についても同様.]
			\item
				$\varphi\colon T_pM\to T_{\tilde{p}}\tilde{M}$を内積を保つ線型同型写像とする.
				任意に与えたなめらかな曲線$c\colon[a,b]\to B_r(0)\subset T_pM$に対し
				\begin{equation}
					\omega(s)=\exp_p(c(s)),\qquad
					\omega_0(s)=\exp_{p_0}(\varphi\circ c(s)),\qquad
					a\leqq s\leqq b
				\end{equation}
				と定める.
				そのとき$L(\omega)\leqq L(\omega_0)$であることを示せ.
		\end{enumerate-subproblems}

		[コメント:これと似たアイディアに基づき,次のような\emph{体積比較定理}を導くこともできる.
		$n$次元完備Riemann多様体$(M,g)$について,ある$k\in\mathbb{R}$が存在して
		$(M,g)$のRicciテンソルが各点で$\Ric\geqq (n-1)k$をみたすとき
		\footnote{Ricciテンソルは各点で$T_pM$上の対称双線型形式を定めるが,
		その最小固有値が$\geqq (n-1)k$であるという意味.},
		定断面曲率$k$をもつ$n$次元単連結完備Riemann多様体\footnote{$k>0$なら半径$1/\sqrt{k}$の球面,
		$k=0$ならEuclid空間,$k<0$なら双曲空間を適当に相似拡大または縮小して断面曲率が$k$になるようにしたもの.}に
		おける半径$r$の球の体積を$\Vol(B^k_r)$と書けば,
		$(M,g)$の任意の点$p$を中心とする距離球$B^d_r(p)$の体積$\Vol(B^d_r(p))$は
		\begin{equation}
			\Vol(B^d_r(p))\leqq\Vol(B^k_r)
		\end{equation}
		をみたす(\emph{Bishopの比較定理}).
		これはさらに
		「$\Vol(B^d_r(p))/\Vol(B^k_r)$は$r$について単調減少である」という形で精密化される
		(\emph{Bishop--Gromovの比較定理}).]
\end{enumerate-problems}

\end{document}
